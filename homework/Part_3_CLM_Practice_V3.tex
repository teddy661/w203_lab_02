% Options for packages loaded elsewhere
\PassOptionsToPackage{unicode}{hyperref}
\PassOptionsToPackage{hyphens}{url}
%
\documentclass[
]{article}
\usepackage{amsmath,amssymb}
\usepackage{lmodern}
\usepackage{iftex}
\ifPDFTeX
  \usepackage[T1]{fontenc}
  \usepackage[utf8]{inputenc}
  \usepackage{textcomp} % provide euro and other symbols
\else % if luatex or xetex
  \usepackage{unicode-math}
  \defaultfontfeatures{Scale=MatchLowercase}
  \defaultfontfeatures[\rmfamily]{Ligatures=TeX,Scale=1}
\fi
% Use upquote if available, for straight quotes in verbatim environments
\IfFileExists{upquote.sty}{\usepackage{upquote}}{}
\IfFileExists{microtype.sty}{% use microtype if available
  \usepackage[]{microtype}
  \UseMicrotypeSet[protrusion]{basicmath} % disable protrusion for tt fonts
}{}
\makeatletter
\@ifundefined{KOMAClassName}{% if non-KOMA class
  \IfFileExists{parskip.sty}{%
    \usepackage{parskip}
  }{% else
    \setlength{\parindent}{0pt}
    \setlength{\parskip}{6pt plus 2pt minus 1pt}}
}{% if KOMA class
  \KOMAoptions{parskip=half}}
\makeatother
\usepackage{xcolor}
\IfFileExists{xurl.sty}{\usepackage{xurl}}{} % add URL line breaks if available
\IfFileExists{bookmark.sty}{\usepackage{bookmark}}{\usepackage{hyperref}}
\hypersetup{
  pdftitle={HW12: CLM Practice},
  pdfauthor={The Principal Components - Ed Brown, Daphne Lin, Linh Tran, Lisa Wu},
  hidelinks,
  pdfcreator={LaTeX via pandoc}}
\urlstyle{same} % disable monospaced font for URLs
\usepackage[margin=1in]{geometry}
\usepackage{color}
\usepackage{fancyvrb}
\newcommand{\VerbBar}{|}
\newcommand{\VERB}{\Verb[commandchars=\\\{\}]}
\DefineVerbatimEnvironment{Highlighting}{Verbatim}{commandchars=\\\{\}}
% Add ',fontsize=\small' for more characters per line
\usepackage{framed}
\definecolor{shadecolor}{RGB}{248,248,248}
\newenvironment{Shaded}{\begin{snugshade}}{\end{snugshade}}
\newcommand{\AlertTok}[1]{\textcolor[rgb]{0.94,0.16,0.16}{#1}}
\newcommand{\AnnotationTok}[1]{\textcolor[rgb]{0.56,0.35,0.01}{\textbf{\textit{#1}}}}
\newcommand{\AttributeTok}[1]{\textcolor[rgb]{0.77,0.63,0.00}{#1}}
\newcommand{\BaseNTok}[1]{\textcolor[rgb]{0.00,0.00,0.81}{#1}}
\newcommand{\BuiltInTok}[1]{#1}
\newcommand{\CharTok}[1]{\textcolor[rgb]{0.31,0.60,0.02}{#1}}
\newcommand{\CommentTok}[1]{\textcolor[rgb]{0.56,0.35,0.01}{\textit{#1}}}
\newcommand{\CommentVarTok}[1]{\textcolor[rgb]{0.56,0.35,0.01}{\textbf{\textit{#1}}}}
\newcommand{\ConstantTok}[1]{\textcolor[rgb]{0.00,0.00,0.00}{#1}}
\newcommand{\ControlFlowTok}[1]{\textcolor[rgb]{0.13,0.29,0.53}{\textbf{#1}}}
\newcommand{\DataTypeTok}[1]{\textcolor[rgb]{0.13,0.29,0.53}{#1}}
\newcommand{\DecValTok}[1]{\textcolor[rgb]{0.00,0.00,0.81}{#1}}
\newcommand{\DocumentationTok}[1]{\textcolor[rgb]{0.56,0.35,0.01}{\textbf{\textit{#1}}}}
\newcommand{\ErrorTok}[1]{\textcolor[rgb]{0.64,0.00,0.00}{\textbf{#1}}}
\newcommand{\ExtensionTok}[1]{#1}
\newcommand{\FloatTok}[1]{\textcolor[rgb]{0.00,0.00,0.81}{#1}}
\newcommand{\FunctionTok}[1]{\textcolor[rgb]{0.00,0.00,0.00}{#1}}
\newcommand{\ImportTok}[1]{#1}
\newcommand{\InformationTok}[1]{\textcolor[rgb]{0.56,0.35,0.01}{\textbf{\textit{#1}}}}
\newcommand{\KeywordTok}[1]{\textcolor[rgb]{0.13,0.29,0.53}{\textbf{#1}}}
\newcommand{\NormalTok}[1]{#1}
\newcommand{\OperatorTok}[1]{\textcolor[rgb]{0.81,0.36,0.00}{\textbf{#1}}}
\newcommand{\OtherTok}[1]{\textcolor[rgb]{0.56,0.35,0.01}{#1}}
\newcommand{\PreprocessorTok}[1]{\textcolor[rgb]{0.56,0.35,0.01}{\textit{#1}}}
\newcommand{\RegionMarkerTok}[1]{#1}
\newcommand{\SpecialCharTok}[1]{\textcolor[rgb]{0.00,0.00,0.00}{#1}}
\newcommand{\SpecialStringTok}[1]{\textcolor[rgb]{0.31,0.60,0.02}{#1}}
\newcommand{\StringTok}[1]{\textcolor[rgb]{0.31,0.60,0.02}{#1}}
\newcommand{\VariableTok}[1]{\textcolor[rgb]{0.00,0.00,0.00}{#1}}
\newcommand{\VerbatimStringTok}[1]{\textcolor[rgb]{0.31,0.60,0.02}{#1}}
\newcommand{\WarningTok}[1]{\textcolor[rgb]{0.56,0.35,0.01}{\textbf{\textit{#1}}}}
\usepackage{graphicx}
\makeatletter
\def\maxwidth{\ifdim\Gin@nat@width>\linewidth\linewidth\else\Gin@nat@width\fi}
\def\maxheight{\ifdim\Gin@nat@height>\textheight\textheight\else\Gin@nat@height\fi}
\makeatother
% Scale images if necessary, so that they will not overflow the page
% margins by default, and it is still possible to overwrite the defaults
% using explicit options in \includegraphics[width, height, ...]{}
\setkeys{Gin}{width=\maxwidth,height=\maxheight,keepaspectratio}
% Set default figure placement to htbp
\makeatletter
\def\fps@figure{htbp}
\makeatother
\setlength{\emergencystretch}{3em} % prevent overfull lines
\providecommand{\tightlist}{%
  \setlength{\itemsep}{0pt}\setlength{\parskip}{0pt}}
\setcounter{secnumdepth}{-\maxdimen} % remove section numbering
\ifLuaTeX
  \usepackage{selnolig}  % disable illegal ligatures
\fi

\title{HW12: CLM Practice}
\author{The Principal Components - Ed Brown, Daphne Lin, Linh Tran, Lisa
Wu}
\date{}

\begin{document}
\maketitle

\hypertarget{part-2---clm-practice}{%
\subsection{Part 2 - CLM Practice}\label{part-2---clm-practice}}

For the following questions, your task is to evaluate the Classical
Linear Model assumptions. It is not enough to say that an assumption is
met or not met; instead, present evidence based on your background
knowledge, visualizations, and numerical summaries.

The file \texttt{videos.txt} contains 9618 observations of videos shared
on YouTube. It was created by Cheng, Dale and Liu at Simon Fraser
University. Please see \href{http://netsg.cs.sfu.ca/youtubedata/}{this
link} for details about how the data was collected.

You wish to run the following regression:

\[
ln(\text{views}) = \beta_0 + \beta_1 \text{rate}  + \beta_3 \text{length}
\] The variables are as follows:

\begin{itemize}
\tightlist
\item
  \texttt{views}: the number of views by YouTube users.
\item
  \texttt{rate}: This is the average of the ratings that the video
  received. You may think of this as a proxy for video quality. (Notice
  that this is different from the variable \texttt{ratings} which is a
  count of the total number of ratings that a video has received.)
\item
  \texttt{length:} the duration of the video in seconds.
\end{itemize}

Response:

\begin{enumerate}
\def\labelenumi{\arabic{enumi}.}
\tightlist
\item
  Evaluate the \textbf{IID} assumption
\end{enumerate}

\begin{itemize}
\item
  Assessing the IID assumption requires an analysis of the sample
  selection design process. Based on our understanding of the selection
  process, the list of videos was selected from YouTube using a crawling
  algorithm which starts with a set of videos from the list of
  ``Recently Featured'', ``Most Viewed'', ``Top Rated'' and ``Most
  Discussed'', for ``Today'', ``This Week'', ``This Month'' and ``All
  Time'' and then uses this list to find more related videos. Given this
  process, we believe that the videos in this dataset are not
  independently sampled. For example, if the sample time frame is around
  election time, we would expect that the initial list of ``Recently
  Featured'' or ``Most Discussed'' videos are more likely to be related
  to the topic of election for ``Today'' or ``This Week''. In addition,
  the crawl algorithm adds videos to the list by finding other videos
  that are directly related to the initial set of videos. Therefore, by
  the nature of the sampling process, this dataset does not meet the IID
  assumption.
\item
  To address this violation of the IID assumption of the classical
  linear model, the researchers need to get new data by using a new
  random sampling process. If the researcher wants to use the same data,
  perhaps a different model type of model other than the classical
  linear model would be more suitable to the current dataset.
\end{itemize}

\begin{enumerate}
\def\labelenumi{\arabic{enumi}.}
\setcounter{enumi}{1}
\item
  Evaluate the \textbf{No perfect Colinearity} assumption.

  In order to assess nearly perfect colinearity, we use our background
  knowledge to evaluate the input variables (rate and length), examine
  the coefficients of the model, review the scatter plot of the two
  variables, and perform correction test and VIF test.

  \begin{itemize}
  \tightlist
  \item
    Based on our background knowledge, the length of a video may affect
    a viewer's rating of the video, but we don't expect near perfect
    correlation between rate and length, as the content of the video
    also plays a key role in viewer's rating.
  \end{itemize}
\end{enumerate}

\begin{Shaded}
\begin{Highlighting}[]
\NormalTok{lm\_video}\SpecialCharTok{$}\NormalTok{coefficients}
\end{Highlighting}
\end{Shaded}

\begin{verbatim}
##  (Intercept)         rate       length 
## 5.4109124371 0.4724853515 0.0004680142
\end{verbatim}

\begin{verbatim}
- We see that R has not dropped rate or length variable in the model which means that there is no perfect colinearity. 

- We examined the scatter plot of rate vs length below which shows that rate and length has no obvious relationship.
\end{verbatim}

\includegraphics[width=0.5\linewidth]{Part_3_CLM_Practice_V3_files/figure-latex/unnamed-chunk-7-1}

\begin{itemize}
\item
  We performed the Pearson correlation test which shows that the
  estimate correlation between rate and length is 0.156 (CI: 0.1372389
  0.1762458, p = 2.2e-16). This means that there is low correlation
  between these two variables.
\item
  We also performed the Tolerance and VIF test below. When Tolerance is
  close to 1 and VIF is less than 5, there is no evidence of the problem
  of nearly perfect collinearity.
\end{itemize}

\begin{table}[!htbp] \centering 
  \caption{Tolerance and VIF Table} 
  \label{} 
\begin{tabular}{@{\extracolsep{5pt}}lccccc} 
\\[-1.8ex]\hline 
\hline \\[-1.8ex] 
Statistic & \multicolumn{1}{c}{N} & \multicolumn{1}{c}{Mean} & \multicolumn{1}{c}{St. Dev.} & \multicolumn{1}{c}{Min} & \multicolumn{1}{c}{Max} \\ 
\hline \\[-1.8ex] 
Tolerance & 2 & 0.9754 & 0.0000 & 0.9754 & 0.9754 \\ 
VIF & 2 & 1.0252 & 0.0000 & 1.0252 & 1.0252 \\ 
\hline \\[-1.8ex] 
\end{tabular} 
\end{table}

\begin{itemize}
\tightlist
\item
  Based on the above assessments, we believe this data meets the
  \textbf{No perfect Colinearity} assumption.
\end{itemize}

\begin{enumerate}
\def\labelenumi{\arabic{enumi}.}
\setcounter{enumi}{2}
\tightlist
\item
  Evaluate the \textbf{Linear Conditional Expectation:} assumption.
\end{enumerate}

\begin{itemize}
\item
  This is an assessment of whether the conditional expectation of Y
  given X exists and has the linear form, which also means the expected
  error term is zero. Based on our background knowledge, views of videos
  are complex observational data and may not simply reflect a linear
  relationship with rate and length of the video. We further plotted the
  residuals (the error term) against each input variable (rate and
  length) and the predicted views to assess whether residuals are close
  to zero with respect to input variables and the outcome variable.
\item
  Looking at the graph of residual versus rate (the left graph below),
  we see that the line of residual average is oscillating around zero
  but is not flat. The line of residual versus length (the middle graph
  below) has a downward curvature (deviate from zero) as length
  increases. These two graphs suggest that there is non-linear
  relationship between the outcome variable (log(views)) and the input
  variables. This non-linear relationship is likely causing the line of
  residuals versus predicted values (the right graph) to curve downward
  as the predicted views increase.
\item
  The evidence presented above shows that the Linear Conditional
  Expectation assumption is not met. Thus, in order to capture the
  relationship between outcome variables and input variable, we may need
  to consider other families of nonlinear models as the linear model
  does not fully model the complexity of the data.
\end{itemize}

\includegraphics[width=0.5\linewidth]{Part_3_CLM_Practice_V3_files/figure-latex/unnamed-chunk-10-1}
\includegraphics[width=0.5\linewidth]{Part_3_CLM_Practice_V3_files/figure-latex/unnamed-chunk-10-2}
\includegraphics[width=0.5\linewidth]{Part_3_CLM_Practice_V3_files/figure-latex/unnamed-chunk-10-3}

\begin{enumerate}
\def\labelenumi{\arabic{enumi}.}
\setcounter{enumi}{3}
\tightlist
\item
  Evaluate the \textbf{Homoskedastic Errors:} assumption.
\end{enumerate}

\begin{itemize}
\tightlist
\item
  To evaluate this assumption, we plotted the residuals against the
  predicted views and evaluated whether the conditional variance is
  constant. From the graph below, residuals start around the range of
  (-4, 4) and widen to the range of (-6, 6) as the log(predicted views)
  increases from around 4.75 to 8. This is strong evidence that the
  conditional variance is not constant.
\end{itemize}

\includegraphics[width=0.5\linewidth]{Part_3_CLM_Practice_V3_files/figure-latex/plotting prediction versus residuals-1}

\begin{itemize}
\tightlist
\item
  Additionally, we performed the Breusch-Pagan (BP) test where the null
  hypothesis is that homoscedasticity is present. With BP value of
  128.39 and p values less than 2.2e-16 means, we reject the null
  hypothesis and conclude that there is strong evidence that
  heteroskedasticity exists in the regression model. Given the
  heteroskedasticity problem, we should use the robust standard error to
  evaluate the fit of the classical linear model. While the problem of
  heteroskedasticity is evident here, one of the key causes could be
  this linear model is not correctly categorizing the relationship
  between input and out variable (as noted in our response to the linear
  conditional assumptions).
\end{itemize}

\begin{verbatim}
## 
##  studentized Breusch-Pagan test
## 
## data:  lm_video
## BP = 128.39, df = 2, p-value < 2.2e-16
\end{verbatim}

\begin{enumerate}
\def\labelenumi{\arabic{enumi}.}
\setcounter{enumi}{4}
\tightlist
\item
  Evaluate the \textbf{Normally Distributed Errors:} assumption.
\end{enumerate}

\begin{itemize}
\tightlist
\item
  From the histogram and Q-Q plots of residuals below, we observe that
  the residuals' distribution has a shape of a normal distribution, with
  slight thin tails (platykurtic). We also measured skewness and
  kurtosis of the errors, with skewness of 0.05738289 (within +-0.5 of
  zero, with zero being normal distribution) and kurtosis of 2.784654
  (within +-0.5 of 3, with 3 being normal distribution), which further
  confirms our observation from the graphs. We conclude that there is no
  strong evidence that this model violates the \textbf{Normally
  Distributed Errors} assumption.
\end{itemize}

\includegraphics[width=0.5\linewidth]{Part_3_CLM_Practice_V3_files/figure-latex/q-q plot of residuals-1}
\includegraphics[width=0.5\linewidth]{Part_3_CLM_Practice_V3_files/figure-latex/q-q plot of residuals-2}
```

\end{document}
